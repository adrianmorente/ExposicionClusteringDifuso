\documentclass[]{report}

\usepackage[utf8]{inputenc}
\usepackage[spanish]{babel}

% Title Page
\title{Agrupamiento difuso}
\author{Carlos Cobos Suárez\\Adrián Morente Gabaldón}


\begin{document}
\maketitle

\begin{abstract}
	HACER RESUMEN
\end{abstract}

	\chapter{Agrupamiento clásico}
	
		\section{Definición}
		
		\section{Conceptos}
		
		\section{Limitaciones}
	
	\chapter{Agrupamiento difuso}
		Tal y como se acaba de ver, el agrupamiento clásico tiene una limitación bastante importante y es la de que un sólo dato puede pertenecer a un \textit{cluster}. Esto también tiene la consecuencia de que, dependiendo del algoritmo de agrupamiento que se vaya a emplear, no se puede garantizar la convergencia del mismo.
		
		A demás, si se usan redes neuronales para llevar a cabo la labor de agrupar los datos, es natural y lógico pensar que un sólo dato va a provocar la activación de más de una neurona. Es por ello que se tiene que cambiar el modelo clásico a uno más permisivo en el cual se solucionen dichos problemas.
			
		\section{Definición}
			
		
		\section{Limitaciones}
	
	\chapter{Agrupamiento difuso posibilístico}
	
		\section{Definición}
	
	\chapter{Aplicaciones reales}
	
\bibliographystyle{plain}
\bibliography{citas}

\end{document}          
